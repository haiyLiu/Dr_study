%% 第二章 组织你的文本

\documentclass[UTF8]{ctexart}


\begin{document}
	''\,‘A’ or ‘B?’\,'' he asked.
	
	%%%% 符号-使用
	% 符号-表示连接
	% 符号--表示数字范围
	% 符号---表示破折号
	An inter-word dash or, hyphen, as in X-ray.
	
	A medium dash for number ranges, like 1--2.
	
	A punctuation dash---like this.
	%%%% -使用
	
	%%%% 省略号使用, \dots和\ldots效果相同
	Good: One, two, three\dots
	
	Bad: One, two, three\ldots
	
	She $\ldots$ she got it.
	
	I've a idea\ldots.
	%%%% 省略号使用
	
	%%%% 特殊符号转义
	\# \quad \$ \quad \% \quad \& \quad
	\{ \quad \} \quad \_ \quad \textbackslash
	%%%% 特殊符号转义
	
	\punctstyle{quanjiao}全角式,所有标点全角,有挤压。例如,“标点挤压”。又如《标点符号用法》。
	
	\punctstyle{banjiao}半角式,所有标点全角,有挤压。例如,“标点挤压”。又如《标点符号用法》。
	
	\punctstyle{kaiming}开明式,部分的标点半角,有挤压。例如,“标点挤压”。又如《标点符号用法》。
	
	\punctstyle{hangmobanjiao}行末半角式,仅行末挤压。例如,“标点挤压”。又如《标点符号用法》。
	
	\punctstyle{plain}无格式,只有禁则,无挤压。例如,“标点挤压”。又如《标点符号用法》。
	
	%%%% 以字母命名的宏
	Happy \TeX ing. Happy \TeX\ ing.
	
	Happy \TeX{} ing. Happy {\TeX} ing.
	%%%% 以字母命名的宏
	
	
	A sentence. And another.
	
	U.S.A means United States Army?
	
	Tinker et al.\ made the double play.
	
	Roman number XII\@. Yes.
	
	
\end{document}