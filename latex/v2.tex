\documentclass[UTF8]{ctexart}
\usepackage{booktabs}

%\usepackage[greek,english]{babel} %%希腊字母和西里尔字母包
%\usepackage[OT2,OT1]{fontenc}{\fontencoding{OT2}\selectfont ABCabc}

\begin{document}
	%% \quad表示间隔
	caf\'e\quad G\"odel\quad Anton\'in Dvo\v{o}\'ak
	
	\O ster Vr\r{o}
	
	K\i rka\u{g}a\c{c}
	
	
	%% 连字,使用{}或\/分开
	dif{}fer f\/ind flight
	
	%% 标点符号使用
	"\,'A' or 'B'?\," he asked. %  ,不能省略,因为\"是另外一种符号的表示,需要加逗号区分
	
	%% 符号-连用情况
	An inter-word dash or, hyphen, as in X-ray. % -连字符
	
	A medium dash for number ranges, like 1--2 or 3$\sim$4. % --表示数字范围
	
	A punctuation dash---like this. % ---表示破折号
	
	Good: One, two, three$\ldots$
	
	中文和English的混排效果并不依赖于 space 的有无。
	
	\CJKsetecglue{} %% 禁止汉字与其他内容之间的空格
	汉字word
	
	幻影\phantom{参数}速速隐形  %%可以用于特殊的占位或者对齐效果
	
	幻影参数速速隐形
	
	
	这是一行文字\\另一行
	
	这是一行文字\linebreak 另一行 \textbackslash %\textbackslash表示反斜线
	
	%% 字体
		% \textit{text} 斜体
		% \textbf{text} 加宽加粗
		% \textnormal{text} 恢复成罗马字
		% \textrm{text} 罗马字
	\textit{Italic font test}
	{\bfseries Bold font test}
	
	\textrm{Roman font family}
	
	\textsf{Sans serif font family}
	
	\ttfamily{Typewriter font family}
	
	\sffamily
	\textbf{This is a paragraph of bold and \textit{itlic font, sometimes returning to \textnormal{normal font} is necessary.}}
	
	Bold '{\bfseries leaf}'
	
	Bold '{\bfseries leaf\/}'
	
	Bold '\textbf{leaf}'
	
	%% 中文字体
	{\heiti 这是黑体}
	
	{\kaishu 这是楷书}
	
	{\fangsong 这是仿宋}
	
	\begin{table}
		\begin{center}
			\caption{中文字号}
			\begin{tabular}{lcl}
				\toprule
				命令 & 大小(bp) & 意义 \\
				\midrule
				\zihao{0} & 42 & \zihao{0}初号 \\
				\bottomrule
			\end{tabular}
		\end{center}
	\end{table}
	
\end{document}