\documentclass{report}
\usepackage{ctex}
\usepackage{booktabs}

\title{Languages\thanks{本文由xxx赞助}}  %%通常用来表示文章的致谢、文档的版本、作者的详细信息等
\author{张三\thanks{讲师}\\九章学堂 \and 李四\\天元研究所}
\date{庚寅盛夏}  %% 可以省略,如果省略,就相当于 \date{\today}
%\date{\CTEXoptions[today=big]}

\begin{document}
	\maketitle   %%% 在 article 或 ctexart 文档类中,title不单独成页;在 report, book 或 ctexrep, ctexbook 文档类中,title单独占用一页
	\tableofcontents	%%% 显示目录  
	
	
	\part{标题和标题页}
		\begin{table}
			\begin{center}  %% 在文中的位置居中
				\caption{章节层次}
				\begin{tabular}{rlll}  %% 如果想让表格有竖线,就写成{r|l|l|l}
					\toprule
					层次 & 名称 & 命令 & 说明 \\
					\midrule
					%% \verb|text|可以使latex保留的命令以文本显示
					-1 & part(部分) & \verb|\part| & 可选的最高层 \\
					
					0 & chapter(章) & \verb|\chapter| & report, book 或 ctexrep, ctexbook文档类的最高层 \\
					
					1 & section(节) & \verb|\section| & article 或 ctexart 类最高层 \\
					
					2 & subsection(小节) & \verb|\subsection| & {} \\
					
					3 & subsubsection(小小节) & \verb|\subsection| & report, book 或 ctexrep, ctexbook 类,默认不编号,不编目录 \\
					
					4 & paragraph(段) & \verb|\paragraph| & 默认不编号、不编目录 \\
					
					5 & subparagraph(小段) & \verb|\subparagraph| & 默认不编号、不编目录 \\
					\bottomrule
				\end{tabular}
			\end{center}
		\end{table}
		
		
	%% 注意:在 \part 下面,\chapter 或 \section 是连续编号的;在其他情况下,下一级的章节随上一节的编号增加会清零重新编号。
	\part{Introduction}							% Part I
		\chapter{Background}					% 	Chapter 1
	\part{Classfication}						% Part II
		\chapter{Natural Language}				% 	Chapter 2
		\chapter{Computer Languages}			% 	Chapter 3
			\section{Machine Languages}			% 		3.1
			\section{High Level Languages}		%		3.2
				\subsection{Compiled Languages}	%			3.2.1
					\paragraph{Common Lisp}
					\paragraph{Schema}
					
					
		%%正文中使用完整的长标题,目录和页眉使用短标题。
		\chapter[展望未来]{展望未来:畅想新时代的计算机排版软件}  
		
	\appendix
	%% 命令 \appendix 后面的所有章(对于 book、report 等)或节(对于 article 等)都将改用字母进行编号:如编号的“Chapter 1” (中文文档类为“第一章”)改为“Appendix A”(中文文档类为“附录 A”)
	\chapter{习题解答}
	
\end{document}